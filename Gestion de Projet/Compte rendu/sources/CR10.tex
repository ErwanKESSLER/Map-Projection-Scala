\documentclass{article}
\usepackage[utf8]{inputenc}
\usepackage[singlespacing]{setspace}
\usepackage{geometry}
\geometry{top=30, bottom=5, left=50, right=50}
\usepackage{array}
\usepackage[T1]{fontenc}
\usepackage[french]{babel}
\usepackage{graphicx}
\usepackage{array,multirow,makecell}
\usepackage{lmodern}
\usepackage{textcomp}
\setcellgapes{1pt}
\makegapedcells
\usepackage[table]{xcolor}
\newcolumntype{R}[1]{>{\raggedleft\arraybackslash }b{#1}}
\newcolumntype{L}[1]{>{\raggedright\arraybackslash }b{#1}}
\newcolumntype{C}[1]{>{\centering\arraybackslash }b{#1}}

\date{} 
\begin{document}
\begin{figure}
    \centering
    \includegraphics[scale=0.05]{logo_TNCY.png}
    \label{fig:logo_tncy}
    \ref{fig:logo_tncy}
\end{figure}
\title{Projet TOP - Compte-rendu n\,°10}
\maketitle
\vspace*{-1cm}

\begin{tabular}{|c|c|}
  \hline
  Motif/ type de réunion: & Lieu: salle PI \\
  \hline
   Présent(s) (retard/excusés/non excusés): &  Date/heure de début/durée:
 & Équipe-projet: Victor Cour,
                  Erwan Kessler,
                  Camille Coué
 & 13 Janvier 2019/165h/2h\\
  \hline
\end{tabular}


\section{Ordre du jour}

\begin{itemize}
  \item Post mortem
  \item Mise en place du GANTT réel
\end{itemize}

\section{Informations échangées/Bilan}
Mise en place du bilan global du projet et des bilans personnels :
\newline
\begin{itemize}
  \item Au niveau technique le projet a permis au groupe de développer des connaissances dans le langage scala, comme la librairie mutable HashMap spéfique à ce langage.
  \newline
  \item Au niveau de la gestion de projet, chaque membre du groupe a pu appliquer les cours de gestion de projet du MOOC. Ces outils nous ont permis de gagner en efficacité. Le GANTT prévisionnel a servi de support tout au long du projet ainsi que la matrice RACI.
 \newline
 \end{itemize}
  
    Création du GANTT réel :
\newline 
 \begin{itemize} 
  \item La mise en place du GANTT réel nous a montré les étapes du projet qui ont pris plus de temps que prévu. 
  Cela nous a permis de dégager des axes d'amélioration au niveau personnel (méthode de travail, organisation) et cela peut nous aider dans la prédiction des étapes pour de futurs projets.
  \end{itemize}
 
  
\end{document}
