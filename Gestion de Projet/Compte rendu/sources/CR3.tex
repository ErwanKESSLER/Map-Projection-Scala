\documentclass{article}
\usepackage[utf8]{inputenc}
\usepackage[singlespacing]{setspace}
\usepackage{geometry}
\geometry{top=30pt, bottom=5pt, left=50pt, right=50pt}
\usepackage{array}
\usepackage[T1]{fontenc}
\usepackage[french]{babel}
\usepackage{graphicx}
\usepackage{array,multirow,makecell}
\usepackage{lmodern}
\usepackage{textcomp}
\setcellgapes{1pt}
\makegapedcells
\usepackage[table]{xcolor}
\newcolumntype{R}[1]{>{\raggedleft\arraybackslash }b{#1}}
\newcolumntype{L}[1]{>{\raggedright\arraybackslash }b{#1}}
\newcolumntype{C}[1]{>{\centering\arraybackslash }b{#1}}

\date{} 
\begin{document}
\begin{figure}
    \centering
    \includegraphics[scale=0.05]{logo_TNCY.png}
    \label{fig:logo_tncy}
    \ref{fig:logo_tncy}
\end{figure}
\title{Projet TOP - Compte-rendu n\,°3}
\maketitle
\vspace*{-1cm}

\begin{tabular}{|c|c|}
  \hline
  Motif/ type de réunion: réunion technique (interne) & Lieu: salle PI \\
  \hline
   Présent(s) (retard/excusés/non excusés): &  Date/heure de début/durée:\\
  Équipe-projet: Victor Cour,
                  Erwan Kessler,
                  Camille Coué
 & 27 Novembre 2018/16h/1h30 \\
  \hline
\end{tabular}


\section{Ordre du jour}
\begin{itemize}
  \item Correction de l'étape 1 (chargement du contenu)
  \item Comparaison des différentes méthodes de calcul des distances 
  \item Sélection des différentes distances
\end{itemize}

\section{Informations échangées}
Lors de la mise en place de l'étape 1, un problème est détecté. La lecture du fichier ligne par ligne avec une séparation par virgule est impossible car certaines chaînes de caractères sont constituées de virgules. La séparation des éléments serait donc fausse. Il faut donc tenir compte de cette contrainte dans la fonction de chargement des données.
\newline
L'équipe projet comparent ensuite les différentes méthodes utilisées pour le calcul des distances.
Plusieurs méthodes vont être utilisées : la distance de Haversine, la distance utilisant la loi des cosinus, la distance utilisée pour les projections equirectangulaires, 
\section{Remarques/Questions}
Au vue des différentes méthodes pour le calcul des distances, laquelle choisir ?
Faut-il privilégier la véracité de la réponse, ou le temps d'exécution au détriment de quelques mètres.
\section{Décisions}
L'équipe va mettre en place les différents calculs de distances puis sélectionner les plus appropriés.
\section{Actions à suivre/ Todo list}

\begin{tabular}{|C{5cm}|C{3cm}|C{3cm}|}
\hline \rowcolor{lightgray} Description & Responsable & Délai \\
\hline  Mise en place de l'étape 2  & Toute l'équipe projet  & 27 Novembre  \\
\hline 
Mise en place de l'étape 3 & Toute l'équipe projet  & 27 Novembre \\
\hline
\end{tabular}
\paragraph{Date de la prochaine réunion}
4 Décembre 2018
\end{document}
