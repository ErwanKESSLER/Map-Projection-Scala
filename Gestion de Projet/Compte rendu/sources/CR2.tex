\documentclass{article}
\usepackage[utf8]{inputenc}
\usepackage[singlespacing]{setspace}
\usepackage{geometry}
\geometry{top=30, bottom=5, left=50, right=50}
\usepackage{array}
\usepackage[T1]{fontenc}
\usepackage[french]{babel}
\usepackage{graphicx}
\usepackage{array,multirow,makecell}
\usepackage{lmodern}
\usepackage{textcomp}
\setcellgapes{1pt}
\makegapedcells
\usepackage[table]{xcolor}
\newcolumntype{R}[1]{>{\raggedleft\arraybackslash }b{#1}}
\newcolumntype{L}[1]{>{\raggedright\arraybackslash }b{#1}}
\newcolumntype{C}[1]{>{\centering\arraybackslash }b{#1}}

\date{} 
\begin{document}
\begin{figure}
    \centering
    \includegraphics[scale=0.05]{logo_TNCY.png}
    \label{fig:logo_tncy}
    \ref{fig:logo_tncy}
\end{figure}
\title{Projet TOP - Compte-rendu n\,°2}
\maketitle
\vspace*{-1cm}


\begin{tabular}{|c|c|}
  \hline
  Motif/ type de réunion: réunion de chantier (interne) & Lieu: salle PI  \\
  \hline
   Présent(s) (retard/excusés/non excusés): &  Date/heure de début/durée:
 & Équipe-projet: Victor Cour,
                  Erwan Kessler,
                  Camille Coué
 & 20 Novembre 2018/16h/1h \\
  \hline
 \end{tabular}

\section{Ordre du jour}
\begin{itemize}
  \item Création d'un diagramme de GANTT prévisionnel
  \item Elaboration d'une matrice RACI (Responsable, Acteur, Consulté, Informé)
  \item Attribution des tâches à chaque membre de l'équipe
  \item Mise en place d'une matrice SWOT(Strengths, Weaknesses, Opportunities, Threats)
\end{itemize}


\section{Informations échangées}
L'équipe projet a dans un premier temps échangé sur les menaces, les opportunités, les forces et les faiblesses du projet. Ensuite le diagramme de GANTT prévisionnel a été mis en place en estimant la durée des étapes du projet. 
Aprés avoir divisé le projet en tâches, ces dernières ont été réparties aux membres de l'équipe dans une matrice RACI.
Prochaines disponiblilités, prévision de gestion de projet.

\section{Remarques/Questions}
/
\section{Décisions}
Ouverture d'un Trello pour la communication du groupe. Victor et Camille occuperont les rôles de secrétaires et Erwan celui de chef de projet.
\section{Actions à suivre/ Todo list}

\begin{tabular}{|C{5cm}|C{3cm}|C{3cm}|}
\hline \rowcolor{lightgray} Description & Responsable & Délai \\
\hline  Mise en place de l'étape 1  & Toute l'équipe projet  & 27 Novembre  \\
\hline 
Mise en place de l'étape 2 & Toute l'équipe projet  & 27 Novembre \\
\hline
\end{tabular}

\paragraph{Date de la prochaine réunion}
27 Novembre 2018
\end{document}
