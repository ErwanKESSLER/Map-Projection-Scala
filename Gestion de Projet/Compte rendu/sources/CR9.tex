\documentclass{article}
\usepackage[utf8]{inputenc}
\usepackage[singlespacing]{setspace}
\usepackage{geometry}
\geometry{top=30pt, bottom=5pt, left=50pt, right=50pt}
\usepackage{array}
\usepackage[T1]{fontenc}
\usepackage[french]{babel}
\usepackage{graphicx}
\usepackage{array,multirow,makecell}
\usepackage{lmodern}
\usepackage{textcomp}
\setcellgapes{1pt}
\makegapedcells
\usepackage[table]{xcolor}
\newcolumntype{R}[1]{>{\raggedleft\arraybackslash }b{#1}}
\newcolumntype{L}[1]{>{\raggedright\arraybackslash }b{#1}}
\newcolumntype{C}[1]{>{\centering\arraybackslash }b{#1}}

\date{} 
\begin{document}
\begin{figure}
    \centering
    \includegraphics[scale=0.05]{logo_TNCY.png}
    \label{fig:logo_tncy}
    \ref{fig:logo_tncy}
\end{figure}
\title{Projet TOP - Compte-rendu n\,°9}
\maketitle
\vspace*{-1cm}

\begin{tabular}{|c|c|}
  \hline
  Motif/ type de réunion: & Lieu: salle PI \\
  \hline
	Présent(s) (retard/excusés/non excusés): 
	&  Date/heure de début/durée:\\
	Équipe-projet: Victor Cour, Erwan Kessler, Camille Coué
	& 08 Janvier 2019/10h/2h\\
  \hline
\end{tabular}


\section{Ordre du jour}

\begin{itemize}
  \item etape 8
  \item compilation
  \item post mortem
\end{itemize}

\section{Informations échangées}
 Mise en place de l'etape 8 pour les projections cylindriques et coniques par la méthode dite de raytracing a travers une geodesie. Necessité d'un datum geodesique: utilisation de celui de D3.js/geo. utilisation de sbt pour construire le projet.

\section{Décisions}
 Mise en place d'un sbt assembly pour generer un jar. Mise en place d'une petite interface, création de la geodesie pour le centrage et la generation de n'importe quelle image.



\section{Actions à suivre/ Todo list}

\begin{tabular}{|C{5cm}|C{3cm}|C{3cm}|}
\hline \rowcolor{lightgray} Description & Responsable & Délai \\
\hline  Post mortem, reunion de concertation pour la soutenance & Toute l'équipe projet  & 13 Janvier  \\
\hline 
\end{tabular}

\paragraph{Date de la prochaine réunion}
13 Janvier 2019
\end{document}
