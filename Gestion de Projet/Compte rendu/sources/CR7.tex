\documentclass{article}
\usepackage[utf8]{inputenc}
\usepackage[singlespacing]{setspace}
\usepackage{geometry}
\geometry{top=30pt, bottom=5pt, left=50pt, right=50pt}
\usepackage{array}
\usepackage[T1]{fontenc}
\usepackage[french]{babel}
\usepackage{graphicx}
\usepackage{array,multirow,makecell}
\usepackage{lmodern}
\usepackage{textcomp}
\setcellgapes{1pt}
\makegapedcells
\usepackage[table]{xcolor}
\newcolumntype{R}[1]{>{\raggedleft\arraybackslash }b{#1}}
\newcolumntype{L}[1]{>{\raggedright\arraybackslash }b{#1}}
\newcolumntype{C}[1]{>{\centering\arraybackslash }b{#1}}

\date{} 
\begin{document}
\begin{figure}
    \centering
    \includegraphics[scale=0.05]{logo_TNCY.png}
    \label{fig:logo_tncy}
    \ref{fig:logo_tncy}
\end{figure}
\title{Projet TOP - Compte-rendu n\,°7}
\maketitle
\vspace*{-1cm}

\begin{tabular}{|c|c|}
  \hline
  Motif/ type de réunion: & Lieu: salle PI \\
  \hline
   Présent(s) (retard/excusés/non excusés): &  Date/heure de début/durée:\\
  Équipe-projet: Victor Cour,
                  Erwan Kessler,
                  Camille Coué
 & 18 Décembre 2018/16h/1h\\
  \hline
\end{tabular}


\section{Ordre du jour}

Avancemement de l'étape 6 :
\begin{itemize}
  \item recherche de l'image bitmap
  \item comprendre comment projeter les coordonnées d'un aéroport sur le planisphère
\end{itemize}

\section{Informations échangées}
Erwan a trouvé une image bitmap haute résolution d’un planisphère correspondant à la projection cylindrique équidistante c'est à dire la projection équirectangulaire.

\section{Remarques/Questions}
 Notre version de scala (2.12.7) ne permet pas d'utiliser l'image wrapper.
\section{Décisions}
L'image wrapper proposée ne sera pas utilisé, une autre solution va être mise en place pour permettre l'affichage du planisphère.
\section{Actions à suivre/ Todo list}

\begin{tabular}{|C{5cm}|C{3cm}|C{3cm}|}
\hline \rowcolor{lightgray} Description & Responsable & Délai \\
\hline  mettre au point les algorithmes de projection de coordonnées & Toute l'équipe projet  & 21 Décembre  \\
\hline 
Mise en place de l'étape 7 & Toute l'équipe projet  & 21 Décembre \\
\hline
\end{tabular}

\paragraph{Date de la prochaine réunion}
21 Décembre 2018
\end{document}

