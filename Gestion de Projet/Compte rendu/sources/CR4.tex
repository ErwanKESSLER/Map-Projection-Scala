\documentclass{article}
\usepackage[utf8]{inputenc}
\usepackage[singlespacing]{setspace}
\usepackage{geometry}
\geometry{top=30pt, bottom=5pt, left=50pt, right=50pt}
\usepackage{array}
\usepackage[T1]{fontenc}
\usepackage[french]{babel}
\usepackage{graphicx}
\usepackage{array,multirow,makecell}
\usepackage{lmodern}
\usepackage{textcomp}
\setcellgapes{1pt}
\makegapedcells
\usepackage[table]{xcolor}
\newcolumntype{R}[1]{>{\raggedleft\arraybackslash }b{#1}}
\newcolumntype{L}[1]{>{\raggedright\arraybackslash }b{#1}}
\newcolumntype{C}[1]{>{\centering\arraybackslash }b{#1}}

\date{} 
\begin{document}
\begin{figure}
    \centering
    \includegraphics[scale=0.05]{logo_TNCY.png}
    \label{fig:logo_tncy}
    \ref{fig:logo_tncy}
\end{figure}
\title{Projet TOP - Compte-rendu n\,°4}
\maketitle
\vspace*{-1cm}

\begin{tabular}{|c|c|}
  \hline
  Motif/ type de réunion: & Lieu: salle PI \\
  \hline
   Présent(s) (retard/excusés/non excusés): &  Date/heure de début/durée:\\
  Équipe-projet: Victor Cour,
                  Erwan Kessler,
                  Camille Coué
 & 4 Decembre 2018/18h/1h30 \\
  \hline
\end{tabular}


\section{Ordre du jour}
\begin{itemize}
  \item Fin de l'étape 2
  \item Etude de l'étape 3 (Statistiques descriptives sur les données)
\end{itemize}
\section{Informations échangées}
La fonction permettant le chargement des données a été modifiée, une expression régulière permet de prendre en compte le problème des données contenant des virgules.
\newline La méthode choisie pour le calcul de la distance : la plus précise (Havershine)
Mise en place des statistiques descriptives du set de données. 
\section{Remarques/Questions}
Quelle est la meilleure méthode de calcul pour la médiane ? 
Choix entre :
\begin{itemize}
  \item un quick sort des données puis sélection de la médiane 
  \item un quick select de la médiane 
  \item un introselect
  \item une amélioration du quick select avec la médiane des médiane 
\end{itemize}

\section{Décisions}
Envoi des compte-rendus sur le GIT en plus du Trello.

\section{Actions à suivre/ Todo list}

\begin{tabular}{|C{5cm}|C{3cm}|C{3cm}|}
\hline \rowcolor{lightgray} Description & Responsable & Délai \\
\hline  Fin de l'étape 3 & Toute l'équipe projet  & 5 Décembre  \\
\hline 
Mise en place de l'étape 4 & Toute l'équipe projet  & 11 Décembre \\
\hline
\end{tabular}

\paragraph{Date de la prochaine réunion}
11 Décembre 2018
\end{document}
