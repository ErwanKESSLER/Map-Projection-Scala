\documentclass{article}
\usepackage[utf8]{inputenc}
\usepackage[singlespacing]{setspace}
\usepackage{geometry}
\geometry{top=30pt, bottom=5pt, left=50pt, right=50pt}
\usepackage{array}
\usepackage[T1]{fontenc}
\usepackage[french]{babel}
\usepackage{graphicx}
\usepackage{array,multirow,makecell}
\usepackage{lmodern}
\usepackage{textcomp}
\setcellgapes{1pt}
\makegapedcells
\usepackage[table]{xcolor}
\newcolumntype{R}[1]{>{\raggedleft\arraybackslash }b{#1}}
\newcolumntype{L}[1]{>{\raggedright\arraybackslash }b{#1}}
\newcolumntype{C}[1]{>{\centering\arraybackslash }b{#1}}

\date{} 
\begin{document}
\begin{figure}
    \centering
    \includegraphics[scale=0.05]{logo_TNCY.png}
    \label{fig:logo_tncy}
    \ref{fig:logo_tncy}
\end{figure}
\title{Projet TOP - Compte-rendu n\,°8}
\maketitle
\vspace*{-1cm}

\begin{tabular}{|c|c|}
  \hline
  Motif/ type de réunion: & Lieu: salle PI \\
  \hline
   Présent(s) (retard/excusés/non excusés): &  Date/heure de début/durée:\\
  Équipe-projet: Victor Cour,
                  Erwan Kessler,
                  Camille Coué
 & 21 Décembre 2018/16h/3h\\
  \hline
\end{tabular}


\section{Ordre du jour}

\begin{itemize}
  \item fin de l'étape 6 
  \item fin de l'étape 7 (projections conforme et équivalente) 
\end{itemize}

\section{Informations échangées}
Pour chaque projection, les latitudes et longitudes doivent être transformées en coordonnées en fonction de la projection. Mais lors des différentes transformations, il faut utiliser les coordonnées x et y des aéroports préalablement transformées linéairement.
Il faut donc procéder en deux étapes : 
\begin{itemize}
  \item trouver les formules des transformations linéaires pour chaque projection
  \item mettre en place les projections
\end{itemize}

\section{Décisions}
Les projections conformes qui vont être mises en place dans un premier temps sont : 
\begin{itemize}
  \item Mercator 
  \item Lambert conique 
  \item Mercator transverse 
\end{itemize}

Au niveau des projections équivalentes, les suivantes vont être mises en place dans un premiers temps :
\begin{itemize}
  \item Lambert cylindrique 
  \item Gall-Peter
  \item Behrmann 
\end{itemize}

\section{Actions à suivre/ Todo list}

\begin{tabular}{|C{5cm}|C{3cm}|C{3cm}|}
\hline \rowcolor{lightgray} Description & Responsable & Délai \\
\hline  Fin de l'étape 7 & Toute l'équipe projet  & 25 Décembre  \\
\hline 
Mise en place de l'étape 8 & Toute l'équipe projet  & 28 Décembre \\
\hline
\end{tabular}

\paragraph{Date de la prochaine réunion}
08 Janvier 2019
\end{document}

